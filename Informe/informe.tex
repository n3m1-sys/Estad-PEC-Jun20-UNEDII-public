\documentclass[a4paper,5pt,titlepage]{article}

\usepackage[utf8]{inputenc}
\usepackage[spanish]{babel}
\usepackage[T1]{fontenc}
\usepackage{lmodern}
\usepackage{amssymb}
\usepackage{amsmath}
\usepackage{xcolor}
\usepackage{listings}
\usepackage{fancyhdr}
\usepackage[a4paper, total={6in, 8in}]{geometry}

\newcommand{\titulo}{Práctica 1, Ejercicio 1}
\newcommand{\tituloc}{Práctica 1, Ejercicio 1}
\newcommand{\asig}{Estadística}
\newcommand{\asigc}{Estadística}

\pagestyle{fancy}
\fancyhf{}
\rhead{Martín Romera Sobrado}
\lhead{\asigc: \tituloc}
\cfoot{\thepage}

\lstset{
	language=R,
	basicstyle=\ttfamily,
	columns=fullflexible,
	frame=single,
	breaklines=true,
	postbreak=\mbox{\textcolor{red}{$\hookrightarrow$}\space},
}

\begin{document}
	El ejercicio que realizo en la práctica es el \textbf{Ejercicio 1}, usando el lenguaje \texttt{R}. Se omiten los enunciados de los ejercicios para optimizar el espacio.
	\section{Apartado 1}
		El siguiente código obtiene los valores aproximados de $E\{T_1\}$ y $E\{T_2\}$:
		\begin{lstlisting}
# Definimos las constantes
# Numero de simulaciones
B <- 10000
# Tamano de las muestras aleatorias simples
n <- c(10, 100, 1000)
# Valor de theta
theta = 2
# Creamos dos matrices para guardar las diferentes simulaciones que se hagan con las diferentes muestras sobre los estimadores
T1 <- matrix(nrow = B, ncol = length(n))
T2 <- matrix(nrow = B, ncol = length(n))
for (i in 1:B) {
	for (j in 1:length(n)) {
		# Obtenemos una muestra aleatoria simple de n = 10, 100 o 1000, dependiendo de la iteracion del segundo bucle
		muestra <- rexp(n[j], theta)
		# Aplicamos la funcion de T1 sobre la muestra y la guardamos en la posicion i = numero de la simulacion y j = log10(n) de la matriz T1
		T1[i,j] <- 1 / mean(muestra)
		# Aplicamos la funcion de T2 sobre la muestra y la guardamos en la posicion i = numero de la simulacion y j = log10(n) de la matriz T2
		T2[i,j] <- (n[j] - 1) / sum(muestra)
	}
}
# Calculamos el valor aproximado de la esperanza matematica de T1 y T2
ET1 <- apply(T1, 2, mean)
ET2 <- apply(T2, 2, mean)
		\end{lstlisting}
		Los vectores \texttt{ET1} y \texttt{ET2} tienen los valores de $E\{T_1\}$ y $E\{T_2\}$ para muestras de $n=10$, $n=100$ y $n=1000$. Un ejemplo de ejecución del código daría los siguientes resultados:
		\begin{center}
			\begin{tabular}{|c|c|c|c|}
				\hline
				$n=$  & 10 & 100 & 1000 \\
				\hline
				\hline
				$E\{T_1\}$ & $2.218567$ & $2.019762$ & $2.002014$ \\
				\hline
				$E\{T_2\}$ & $1.996710$ & $1.999564$ & $2.000012$ \\
				\hline
			\end{tabular}
		\end{center}
		Podemos ver que los valores aporximados de la esperanza matemática de ambos estimadores se aproximan a $\theta$, más si el tamaño de la muestra crece. Sin embargo, los valores de $E\{T_2\}$ son más cercanos a $\theta$ que los de $E\{T_1\}$.
	\newpage		
	\section{Apartado 2}
		Para la aproximación del sesgo de los estimadores, tras una ejecución del código previo, ejecutamos las siguientes lineas:
		\begin{lstlisting}
bT1 = ET1 - theta
bT2 = ET2 - theta
		\end{lstlisting}
		Siendo \texttt{bT1} y \texttt{bT2}, dos vectores que tienen el valor $b$ (aproximación del sesgo), para los dos estimadores en los diferentes espacios muestrales. Partiendo de los resultados previos, obtenemos los siguientes valores:
		\begin{center}
			\begin{tabular}{|c|c|c|c|}
				\hline
				$n=$  & 10 & 100 & 1000 \\
				\hline
				\hline
				$b_{T_1}$ & $0.218566725$ & $0.0197619925$ & $2.014069\cdot10^{-3}$ \\
				\hline
				$b_{T_2}$ & $-0.003289947$ & $-0.0004356274$ & $1.205515 \cdot10^{-5}$ \\
				\hline
			\end{tabular}
		\end{center}
		Observamos que el sesgo disminuye en ambos estimadores según crece el espacio muestral. También y como habíamos predicho en el apartado anterior, los valores de $E\{T_2\}$ se aproximan más a $\theta$ que los de $E\{T_1\}$, lo que hace que $|b_{T_2}|$ sea menor para cualquiera de los espacios muestrales y el sesgo sea menor y se emplearía el estimador $T_2$. 
\end{document}